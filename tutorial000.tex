\documentclass[a4paper, 11pt]{article}
\usepackage{verbatim}
\usepackage[colorlinks]{hyperref}
\usepackage{graphicx, float}
\graphicspath{{assets/images/}}
% \usepackage{amsmath, amssymb, amsthm, amsfonts}
\usepackage{amsmath}
\usepackage[listings]{tcolorbox}
\tcbset{colback=blue!4!white,colframe=blue!50!green!30!gray}
% Margins
\usepackage[top=1.5cm, bottom=2cm, left=2.4cm, right=1.8cm]{geometry}
% Temporary fix until better understanding of the indentation rules
\setlength{\parindent}{0pt}

\usepackage{fontspec}
\newfontfamily\Nerd{Symbols Nerd Font}[Scale=1.4]
\newfontfamily\DejaSans{DejaVu Sans}[Scale=0.9]
\newfontfamily\Iosevka{Iosevka Custom}[Scale=0.95]
\newfontfamily\Symbola{Symbola}[Scale=1.0]
\newfontfamily\FreeSerif{FreeSerif}[Scale=1.1]
\newfontfamily\Libertine{Linux Libertine O}[Scale=1.1]
\newfontfamily\Biolinum{Linux Biolinum O}[Scale=1.1]

% \setmainfont{FreeSerif}[Scale=1.05]
% \setmainfont{Linux Biolinum O}[Scale=1.1]
% \setmainfont{Linux Libertine O}[Scale=1.1]

\usepackage{fontawesome5}

\title{Yet Another \LaTeX{} Tutorial \\
  \medskip \large Breadcrumbs in the road to Procrastination}
\author{Energos}
\date{\today}

\begin{document}

% \FreeSerif

\maketitle

\section{Why?}
\href{https://www.youtube.com/watch?v=9eLjt5Lrocw}
{Why \LaTeX{}? - YouTube}

\section{Install \LaTeX{} on Debian}
\begin{tcolorbox}
\begin{verbatim}
apt install texlive texlive-science texlive-lang-english \
    texlive-fonts-extra texlive-xetex
\end{verbatim}
\end{tcolorbox}

\section{The Document}
\begin{tcolorbox}
\begin{verbatim}
\begin{document}
\maketitle
\section{Section title}
\subsection{Subsection title}
\end{document}
\end{verbatim}
\end{tcolorbox}

\section{Title}
\begin{tcolorbox}
\begin{verbatim}
\title{Yet Another \LaTeX{} Tutorial \\
       \medskip \large A road to Procrastination}
\author{Energos}
\date{March 2025}
\end{verbatim}
\end{tcolorbox}

\section{A Section}
\begin{tcolorbox}
\begin{verbatim}
\section{A Section}
\end{verbatim}
\end{tcolorbox}

\subsection{A Subsection}
\begin{tcolorbox}
\begin{verbatim}
\section{A Subsection}
\end{verbatim}
\end{tcolorbox}

\section{Generate a PDF file}
\subsection{\TeX{} Engines}
\href{https://www.overleaf.com/learn/latex/Articles/What\%27s_in_a_Name\%3A_A_Guide_to_the_Many_Flavours_of_TeX}
{What's in a Name: A Guide to the Many Flavours of TeX - Overleaf, Online LaTeX Editor}
\subsection{Using the XeTeX engine}
This is the preferred engine so far...
\begin{tcolorbox}
\begin{verbatim}
xelatex thisfile.tex
\end{verbatim}
\end{tcolorbox}
\subsection{Using the pdfTeX engine}
\begin{tcolorbox}
\begin{verbatim}
pdflatex thisfile.tex
\end{verbatim}
\end{tcolorbox}

\section{Paragraph indentation}
Temporary fix until better understanding of the indentation rules.
\begin{tcolorbox}
\begin{verbatim}
\setlength{\parindent}{0pt}
\end{verbatim}
\end{tcolorbox}

\section{Margins}
\begin{tcolorbox}
\begin{verbatim}
\usepackage[top=0.75in, bottom=1.00in, left=1.20in, right=1.00in]{geometry}
\end{verbatim}
\end{tcolorbox}

\section{Lists}
\begin{tcblisting}{}
  \begin{itemize}
  \item One
  \item Two
  \item Three
  \end{itemize}
\end{tcblisting}

\begin{tcblisting}{}
  \begin{enumerate}
  \item One
  \item Two
    \begin{enumerate}
    \item Two One
    \item Two Two
    \end{enumerate}
  \item Three
  \end{enumerate}
\end{tcblisting}

\section{URL links}
\begin{tcolorbox}
\begin{verbatim}
\usepackage[colorlinks]{hyperref}
\end{verbatim}
\end{tcolorbox}

\begin{tcblisting}{}
  \href{https://www.ctan.org/pkg/hyperref}{CTAN: Package hyperref}
\end{tcblisting}

\section{Verbatim}
{\Nerd } Warnings: \\
- Font configuration issues. \\
- Do not break long lines. \\
{\Nerd } TODO: Investigate the package verbatimbox.
\subsection{Using the package verbatim}
\begin{tcolorbox}
\begin{verbatim}
\usepackage{verbatim}
\end{verbatim}
\end{tcolorbox}
\subsubsection{Including verbatim text}
\begin{tcblisting}{}
\begin{verbatim}
Verbatim stuff.
\LaTeX{} keywords are printed verbatim.
More verbatim stuff...
\end{verbatim}
\end{tcblisting}
\subsubsection{Including a file verbatim}
\begin{tcolorbox}
\begin{verbatim}
\verbatiminput{file_name}
\end{verbatim}
\end{tcolorbox}
\subsection{Using \textbackslash{}verb}
\begin{tcblisting}{}
  \verb|Verbatim stuff.| \\
  \verb|\LaTeX{} keywords are printed verbatim.| \\
  \verb|More verbatim stuff...|
\end{tcblisting}
{\Nerd } Warning: \verb|\|verb is illegal in arguments.

\section{Fonts}
{\Nerd } TODO:
\href{https://www.overleaf.com/learn/latex/Font_sizes\%2C_families\%2C_and_styles}
{Font sizes, families, and styles - Overleaf, Online LaTeX Editor} \\
{\Nerd } Warning: verbatim and friends, tcolorbox and tcblisting  do not print the
{\Nerd } and other ``funny'' characters, neither lowercase Greek letters and some
synbols. Probably it is a font configuration issue.

{\Nerd } Use a font that includes all the required glyphs.
\begin{tcolorbox}
\begin{verbatim}
\usepackage{fontspec}

% Pick your favorite fonts
\newfontfamily\DejaSans{DejaVu Sans}
\newfontfamily\Iosevka{Iosevka Custom}
\newfontfamily\Symbola{Symbola}[Scale=1.1]
\end{verbatim}
\end{tcolorbox}

\def \sLorem {
  Lorem ipsum dolor sit amet. Qui officiis repellendus id sunt consequatur
  ut nisi eligendi est rerum dolorem qui rerum porro non laudantium quasi ut
  laudantium aspernatur.

  Sed nesciunt voluptatem eos internos delectus sed blanditiis explicabo nam
  officia voluptatem vel provident minus non distinctio nihil.}
\def \sGreek {Αα Ββ Γγ Δδ Εϵε Ζζ Ηη Θθϑ Ιι Κκ Λλ Μμ Νν Ξξ Οο Ππ Ρρϱ Σσς Ττ Υυ Φϕφ Χχ Ψψ Ωω}
\def \sSymbols {⚠⁉ ℃ ℉ ← ⇐ → ⇒ ↑ ⇑ ↓ ⇓ ↔ ⇔ ↕ ⇕ ∞ ∇ △ √ × ÷ ⊕ ⊗ ≡ ∼ = ≃ ≈ ∑ ∏ ≤ ≥ α β γ Γ δ ∆ ω Ω}

\underline{Default document font}: \fontname\font \\ [0.5ex]
\sLorem \\ [0.5ex] \sGreek \\ [0.5ex] \sSymbols \\ [1.0ex]
\underline{Libertine}: {\Libertine \fontname\font \\ [0.5ex]
  \sLorem \\ [0.5ex] \sGreek \\ [0.5ex] \sSymbols} \\ [1.0ex]
\underline{Biolinum}: {\Biolinum \fontname\font \\ [0.5ex]
  \sLorem \\ [0.5ex] \sGreek \\ [0.5ex] \sSymbols} \\ [1.0ex]
\underline{FreeSerif}: {\FreeSerif \fontname\font \\ [0.5ex]
  \sLorem \\ [0.5ex] \sGreek \\ [0.5ex] \sSymbols} \\ [1.0ex]
\underline{Symbola}: {\Symbola \fontname\font \\ [0.5ex]
  \sLorem \\ [0.5ex] \sGreek \\ [0.5ex] \sSymbols} \\ [1.0ex]
\underline{DejaVu Sans}: {\DejaSans \fontname\font \\ [0.5ex]
  \sLorem \\ [0.5ex] \sGreek \\ [0.5ex] \sSymbols} \\ [1.0ex]
\underline{Iosevka}: {\Iosevka \fontname\font \\ [0.5ex]
  \sLorem \\ [0.5ex] \sGreek \\ [0.5ex] \sSymbols} \\ [1.0ex]
And now, back to \underline{Default document font}: \fontname\font

\section{Font size}
\begin{tabbing}
  {\Huge \LaTeX{} \texttt{Tutorial}}         \=  (\verb|\Huge|) \\
  {\huge \LaTeX{} \texttt{Tutorial}}         \>  (\verb|\huge|) \\
  {\LARGE \LaTeX{} \texttt{Tutorial}}        \>  (\verb|\LARGE|) \\
  {\Large \LaTeX{} \texttt{Tutorial}}        \>  (\verb|\Large|) \\
  {\large \LaTeX{} \texttt{Tutorial}}        \>  (\verb|\large|) \\
  {\normalsize \LaTeX{} \texttt{Tutorial}}   \>  (\verb|\normalsize|) \\
  {\small \LaTeX{} \texttt{Tutorial}}        \>  (\verb|\small|) \\
  {\footnotesize \LaTeX{} \texttt{Tutorial}} \>  (\verb|\footnotesize|) \\
  {\scriptsize \LaTeX{} \texttt{Tutorial}}   \>  (\verb|\scriptsize|) \\
  {\tiny \LaTeX{} \texttt{Tutorial}}         \>  (\verb|\tiny|)
\end{tabbing}

\section{Nerd Fonts}
\href{https://www.nerdfonts.com/}{Nerd Fonts} must be installed in your system.
\begin{tcolorbox}
\begin{verbatim}
\usepackage{fontspec}
\newfontfamily\Nerd{Symbols Nerd Font}[Scale=1.4]
\end{verbatim}
\end{tcolorbox}
\begin{tabbing}
  This is a very small sample: \\  [1.0ex]
  codicon   (Codicons) ~~~~~~~~~~~~~~~  \= {\Nerd              } \\
  devicon   (Devicons)                  \> {\Nerd                 } \\
  faicon    (Font Awesome)              \> {\Nerd              } \\
  flicon    (Font Logos)                \> {\Nerd             } \\
  ipsicon   (IEC Power Symbols)         \> {\Nerd ⏻ ⏽ ⭘ ⏾ ⏼} \\
  mdicon    (Material Design)           \> {\Nerd 󰣚 󰣨 󱄅 󰣠 󱘗 󰙲 󰙱 󰢱 󰌠 󰲒 󰊢 󰸞 󰅖 󰚌 󰊠 󰠭 󰃤} \\
  octicon   (Octicons)                  \> {\Nerd      ♥        } \\
  pomicon   (Pomicons)                  \> {\Nerd       } \\
  powerline (Powerline Symbols)         \> {\Nerd              } \\
  sucicon   (Seti-UI + Custom)          \> {\Nerd                } \\
  wicon     (Weather)                   \> {\Nerd        }
\end{tabbing}

\section{Font Awesome}
On Debian you need to install the package texlive-fonts-extra.
\begin{tcolorbox}
\begin{verbatim}
\usepackage{fontawesome5}
\end{verbatim}
\end{tcolorbox}

\begin{tabbing}
  This is this is an insignificantly small sample: \\ [1.0ex]
  Nerd Fonts (for reference) ~~~~~~~~~  \= {\Nerd              } \\ [1.0ex]
  Font Awesome (normal) \> \faGit{} \faGit*{} \faGithub{} \faYCombinator{} \faCheck{} \faTimes{} \faBug{} \faPoop{} \faSkull{} \faSkullCrossbones{} \faArrowLeft{} \faArrowUp{} \faArrowRight{} \faArrowDown{} \\
  Font Awesome (large) \> \large \faGit{} \faGit*{} \faGithub{} \faYCombinator{} \faCheck{} \faTimes{} \faBug{} \faPoop{} \faSkull{} \faSkullCrossbones{} \faArrowLeft{} \faArrowUp{} \faArrowRight{} \faArrowDown{} \\
  Font Awesome (Large)\> \Large \faGit{} \faGit*{} \faGithub{} \faYCombinator{} \faCheck{} \faTimes{} \faBug{} \faPoop{} \faSkull{} \faSkullCrossbones{} \faArrowLeft{} \faArrowUp{} \faArrowRight{} \faArrowDown{} \\
  Font Awesome (LARGE) \> \LARGE \faGit{} \faGit*{} \faGithub{} \faYCombinator{} \faCheck{} \faTimes{} \faBug{} \faPoop{} \faSkull{} \faSkullCrossbones{} \faArrowLeft{} \faArrowUp{} \faArrowRight{} \faArrowDown{}
\end{tabbing}


\section{Greek Letters and Math Symbols}

\begin{tcblisting}{listing side text, righthand width=5cm}
  \begin{tabbing}
    Α ~ \= Alpha ~~~~~ \= $A \alpha$ \\
    Β   \> Beta      \> $B \beta$ \\
    Γ   \> Gamma     \> $\Gamma \gamma$ \\
    Δ   \> Delta     \> $\Delta \delta$ \\
    Ε   \> Epsilon   \> $E \epsilon \varepsilon$ \\
    Ζ   \> Zeta      \> $Z \zeta$ \\
    Η   \> Eta       \> $H \eta$ \\
    Θ   \> Theta     \> $\Theta \theta \vartheta$ \\
    Ι   \> Iota      \> $I \iota$ \\
    Κ   \> Kappa     \> $K \kappa$ \\
    Λ   \> Lambda    \> $\Lambda \lambda$ \\
    Μ   \> Mu        \> $M \mu$ \\
    Ν   \> Nu        \> $N \nu$ \\
    Ξ   \> Xi        \> $\Xi \xi$ \\
    Ο   \> Omicron   \> $O o$ \\
    Π   \> Pi        \> $\Pi \pi$ \\
    Ρ   \> Rho       \> $P \rho \varrho$ \\
    Σ   \> Sigma     \> $\Sigma \sigma$ \\
    Τ   \> Tau       \> $T \tau$ \\
    Υ   \> Upsilon   \> $\Upsilon \upsilon$ \\
    Φ   \> Phi       \> $\Phi \phi \varphi$ \\
    Χ   \> Chi       \> $X \chi$ \\
    Ψ   \> Psi       \> $\Psi \psi$ \\
    Ω   \> Omega     \> $\Omega \omega$
  \end{tabbing}
\end{tcblisting}
{\Nerd } \>
Math Mode: $A \alpha \; B \beta \; \Gamma \gamma \; \Delta \delta \; \cdots \; X \chi \; \Psi \psi \; \Omega \omega$ \>\>\>\>
Text Mode: {\Symbola \large Aα\, Bβ\, Γγ\, Δδ\, \textperiodcentered \textperiodcentered \textperiodcentered \, Xχ\, Ψψ\, Ωω}

\bigskip
\textbf{Math symbols}, a small sample:
\begin{tabbing}
  \underline{Keyword} ~~~~~~~~~~~~~ \= \underline{Symbol} ~~~~~~~ \=
  \underline{Keyword} ~~~~~~~~~~~~~ \= \underline{Symbol} ~~~~~~~ \=
  \underline{Keyword} ~~~~~~~~~~~~~ \= \underline{Symbol} \\ [0.5ex]
  \verb|\leftarrow|       \> $\leftarrow$       \> \verb|\Leftarrow|       \> $\Leftarrow$       \> \verb|\oplus|      \> $\oplus$      \\
  \verb|\rightarrow|      \> $\rightarrow$      \> \verb|\Rightarrow|      \> $\Rightarrow$      \> \verb|\otimes|     \> $\otimes$     \\
  \verb|\uparrow|         \> $\uparrow$         \> \verb|\Uparrow|         \> $\Uparrow$         \> \verb|\odot|       \> $\odot$       \\
  \verb|\downarrow|       \> $\downarrow$       \> \verb|\Downarrow|       \> $\Downarrow$       \> \verb|\bigoplus|   \> $\bigoplus$   \\
  \verb|\leftrightarrow|  \> $\leftrightarrow$  \> \verb|\Leftrightarrow|  \> $\Leftrightarrow$  \> \verb|\bigotimes|  \> $\bigotimes$  \\
  \verb|\updownarrow|     \> $\updownarrow$     \> \verb|\Updownarrow|     \> $\Updownarrow$     \> \verb|\bigodot|    \> $\bigodot$    \\
  \verb|\equiv|           \> $\equiv$           \> \verb|\leq|             \> $\leq$             \> \verb|\geq|        \> $\geq$        \\
  \verb|\neq|             \> $\neq$             \> \verb|\ll|              \> $\ll$              \> \verb|\gg|         \> $\gg$         \\
  \verb|\sim|             \> $\sim$             \> \verb|\simeq|           \> $\simeq$           \> \verb|\infty|      \> $\infty$      \\
  \verb|\approx|          \> $\approx$          \> \verb|\cong|            \> $\cong$            \> \verb|\propto|     \> $\propto$     \\
  \verb|\nabla|           \> $\nabla$           \> \verb|\times|           \> $\times$           \> \verb|\sum|        \> $\sum$        \\
  \verb|\triangle|        \> $\triangle$        \> \verb|\div|             \> $\div$             \> \verb|\prod|       \> $\prod$
\end{tabbing}

\section{Images}
\begin{tcolorbox}
\begin{verbatim}
\usepackage{graphicx, float}
\graphicspath{{assets/images/}}
\end{verbatim}
\end{tcolorbox}

\begin{tcblisting}{}
  \begin{figure}[H]
    \centering
    \includegraphics[width=4in]{latex-project.png}
    \caption{\LaTeX{}}
    \label{fig:latex-project-logo}
  \end{figure}
\end{tcblisting}

\begin{tcblisting}{}
  \begin{figure}[H]
    \centering
    \href{https://devuan.org}
    {\includegraphics[width=4in]{devuan-logo_tm_600dpi.png}}
    \caption{Image acts as a hyperlink to a URL}
    \label{fig:devuan-logo}
  \end{figure}
\end{tcblisting}

\bigskip
SVG Images require more work:
\href{https://www.baeldung.com/cs/latex-svg-images}
{How to Use SVG Images in \LaTeX{}}.

\section{Math Mode}
\subsection{Inline Math Mode}
Math Mode: \verb|$...$|
\begin{tcblisting}{}
  This is a meaningless equation: $y=x^0+x^1+x^2+x^3+x^4+x^5+x^6+x^7+x^8$ \\
  Extra      spaces     are    irrelevant. \\
  This   is   a  meaningless  equation:   $ y =x ^ 0 + x ^ 1 + x ^ 2 + x ^ 3 + x ^ 4 + x ^ 5 + x ^ 6 + x ^ 7 + x ^ 8 $
\end{tcblisting}{}

\subsection{Protected inline Math Mode}
Protected Math Mode: \verb|${...}$|
\begin{tcblisting}{}
  Not protected: \\
  This is a very long line with a very long meaningless equation $y=x^0+x^1+x^2+x^3+x^4+x^5+x^6+x^7+x^8$ that can be broken. \\
  Protected: \\
  This is a very long line with a very long meaningless equation ${y=x^0+x^1+x^2+x^3+x^4+x^5+x^6+x^7+x^8}$ that will not be broken.
\end{tcblisting}

{\Nerd } Warning: Protected inline Math Mode is buggy {\scriptsize \Nerd } when not inside a box or a minipage. \\
This is a very long line with a very long meaningless equation ${y=x^0+x^1+x^2+x^3+x^4+x^5+x^6+x^7+x^8}$ that will not be broken.
This is clearly not working properly.

\subsection{Display Math Mode}
Display Math Mode: \verb|$$...$$| or \verb|\[...\]|
\begin{tcblisting}{}
  $$y=x^0+x^1+x^2+x^3+x^4$$ or \[y=x^0+x^1+x^2+x^3+x^4\]
  Extra   spaces   are   irrelevant.   $$ y =x ^ 0 + x ^ 1 + x ^ 2 + x ^ 3 + x ^ 4 $$
\end{tcblisting}

\section{Equations}

\begin{tcolorbox}
\begin{verbatim}
% \usepackage{amsmath, amssymb, amsthm, amsfonts}
\usepackage{amsmath}
\end{verbatim}
\end{tcolorbox}


\begin{tcblisting}{}
  \begin{tcolorbox}[title=Maxwell Equations]
    \begin{align}
      \nabla \cdot \mathbf{E} \,\,\, &= \frac{\rho}{\varepsilon_0} \\
      \nabla \cdot \mathbf{B} \,\,\, &= 0 \\
      \nabla \times \mathbf{E} &= -\frac{\partial \mathbf{B}}{\partial t} \\
      \nabla \times \mathbf{B} &= \mu_0 \left(\mathbf{J}
                                 + \varepsilon_0 \frac{\partial
                                 \mathbf{E}}{\partial t} \right)
    \end{align}
  \end{tcolorbox}
\end{tcblisting}

\begin{tcblisting}{}
  \begin{align}
    \sin x &= \sum^{\infty}_{n=0} \frac{(-1)^n}{(2n+1)!} x^{2n+1}
    &= x - \frac{x^3}{3!} + \frac{x^5}{5!} - \frac{x^7}{7!} + \frac{x^9}{9!}
      - \cdots && \text{for all } x\\[6pt]
    \cos x &= \sum^{\infty}_{n=0} \frac{(-1)^n}{(2n)!} x^{2n}
    &= 1 - \frac{x^2}{2!} + \frac{x^4}{4!} - \frac{x^6}{6!} + \frac{x^8}{8!}
      - \cdots && \text{for all } x
  \end{align}
\end{tcblisting}

\section{Simple Tab Stops}
\begin{tcblisting}{}
  \begin{tabbing}
    ~~~~~~~~~~~~~~ \= ~~~~~~~~~~~~~~ \= 0 \\
    1 \> 1 \> 1 \\
    22 \> 22 \> 22 \\
    333 \> 333 \> 333 \\
    4444 \> 4444 \> 4444
  \end{tabbing}
\end{tcblisting}

\section{Tables}
\begin{tcblisting}{}
  \begin{tabular}{lll}
    & & 0 \\
    1 & 1 & 1 \\
    22 & 22 & 22 \\
    333 & 333 & 333 \\
    4444 & 4444 & 4444
  \end{tabular}
\end{tcblisting}

\begin{tcblisting}{}
  \begin{table}[H]
    \def\arraystretch{1.2}
    \centering
    % c "centered", l "left", r "right"
    \begin{tabular}{|c||r|r|r|r|r|}
      \hline
      $x$ & 1 & 2 & 3 & 4 & 5 \\ \hline
      $y=x^2$ & 1 & 4 & 9 & 16 & 25 \\
      $y=x^3$ & 1 & 8 & 27 & 64 & 125 \\ \hline
    \end{tabular}
    \caption{Squares and Cubes}
  \end{table}
\end{tcblisting}

\section{Boxes}
Draw colored and framed boxes using
\href{https://ctan.org/pkg/tcolorbox?lang=en}{tcolorbox}.
This package has a huge set of options and features.
Please read the
\href{http://mirrors.ctan.org/macros/latex/contrib/tcolorbox/tcolorbox.pdf}
{documentation} and
\href{http://mirrors.ctan.org/macros/latex/contrib/tcolorbox/tcolorbox-example.pdf}
{examples of use}.

\begin{tcolorbox}
\begin{verbatim}
\usepackage{tcolorbox}
% Define document wide default colors
\tcbset{colback=blue!4!white,colframe=blue!50!green!30!gray}
\end{verbatim}
\end{tcolorbox}

\subsection{Simple Boxes}

\begin{tcolorbox}
\begin{verbatim}
\begin{tcolorbox}
  This is a box using document wide default colors.
\end{tcolorbox}
\end{verbatim}
\end{tcolorbox}
\begin{tcolorbox}
  This is a box using document wide default colors.
\end{tcolorbox}

\bigskip

\begin{tcolorbox}
\begin{verbatim}
\begin{tcolorbox}[colback=red!30!white,colframe=red!80!black]
  This is a somewhat reddish box.
\end{tcolorbox}
\end{verbatim}
\end{tcolorbox}

\begin{tcolorbox}[colback=red!30!white,colframe=red!80!black]
  This is a somewhat reddish box.
\end{tcolorbox}

\subsection{Listing Boxes}

\begin{tcolorbox}
\begin{verbatim}
\usepackage[listings]{tcolorbox}

\begin{tcblisting}{colback=red!5!white,colframe=red!75!black}
  Boxes inside a box:
  \begin{tcolorbox}
    This is a box using document wide default colors.
  \end{tcolorbox}
  \begin{tcolorbox}[colback=red!30!white,colframe=red!80!black]
    This is a somewhat reddish box.
  \end{tcolorbox}
\end{tcblisting}
\end{verbatim}
\end{tcolorbox}

\begin{tcblisting}{colback=red!5!white,colframe=red!75!black}
  Boxes inside a box:
  \begin{tcolorbox}
    This is a box using default colors.
  \end{tcolorbox}
  \begin{tcolorbox}[colback=red!30!white,colframe=red!80!black]
    This is a somewhat reddish box.
  \end{tcolorbox}
\end{tcblisting}

\bigskip

\begin{tcolorbox}
\begin{verbatim}
\begin{tcblisting}{colback=red!5!white,colframe=red!75!black}
  This is a \LaTeX\ example:
  \begin{equation*}
    \sum\limits_{i=1}^n i = \frac{n(n+1)}{2}.
  \end{equation*}
\end{tcblisting}
\end{verbatim}
\end{tcolorbox}

\begin{tcblisting}{colback=red!5!white,colframe=red!75!black}
  This is a \LaTeX\ example:
  \begin{equation*}
    \sum\limits_{i=1}^n i = \frac{n(n+1)}{2}.
  \end{equation*}
\end{tcblisting}

\section{Packages}
- \href{https://www.ctan.org/pkg/hyperref}{CTAN: Package hyperref} \\
- \href{https://ctan.org/pkg/verbatim}{CTAN: Package verbatim} \\
- \href{https://ctan.org/pkg/geometry}{CTAN: Package geometry} \\
- \href{https://www.ctan.org/pkg/graphicx}{CTAN: Package graphicx} \\
- \href{https://www.ctan.org/pkg/float}{CTAN: Package float} \\
- \href{https://ctan.org/pkg/amsmath?lang=en}{CTAN: Package amsmath} \\
- \href{https://ctan.org/pkg/tcolorbox?lang=en}{CTAN: Package tcolorbox} \\
- \href{https://ctan.org/pkg/fontawesome5?lang=en}{CTAN: Package fontawesome5}

\section{Resources}
- \href{https://www.latex-project.org/}
{LaTeX - A document preparation system} \\
- \href{https://www.ctan.org/}{CTAN: Comprehensive TeX Archive Network} \\
- \href{https://texdoc.org/index.html}
{texdoc - TeX and LaTeX documentation lookup system} \\
- \href{https://www.ams.org/arc/resources/amslatex-about.html}
{American Mathematical Society - LaTeX extensions} \\
- \href{https://www.ams.org/arc/resources/texresources.html}
{American Mathematical Society - TeX Resources} \\
- \href{https://texfaq.org/}
{The TeX Frequently Asked Question List - The TeX FAQ} \\
- \href{https://texfaq.org/FAQ-latex-books}{Books on LaTeX - The TeX FAQ} \\
- \href{https://www.tug.org/FontCatalogue/}{The LaTeX Font Catalogue} \\
- \href{https://tug.ctan.org/info/symbols/comprehensive/symbols-a4.pdf}
{The Comprehensive LaTeX Symbol List - CTAN} \\
- \href{https://www.learnlatex.org/en/}
{Learn LaTeX online for free in beginner friendly lessons | learnlatex.org} \\
- \href{https://r2src.github.io/top10fonts/}{Top 10 LaTeX Fonts} \\
- \href{https://www.draketo.de/anderes/latex-fonts.html}{Top 20 Latex Fonts} \\
- \href{https://www.fontspace.com/unicode/char/26A0-warning-sign}
{“{\Symbola ⚠}” WARNING SIGN | U+26A0 Unicode | FontSpace} \\
- \href{https://www.gnu.org/software/auctex/}
{AUCTeX - Sophisticated document creation in GNU Emacs}

\section{Tutorials}
- \href{https://www.youtube.com/watch?v=lgiCpA4zzGU}
{LaTeX for Students – A Simple Quickstart Guide - YouTube} \\
- \href{https://www.youtube.com/watch?v=ydOTMQC7np0}
{LaTeX - Full Tutorial for Beginners - YouTube} \\
- \href{https://www.youtube.com/watch?v=VqMETR1GeYI}
{LaTeX Basics 00: Installation and Hello, World! - YouTube} \\
- \href{https://www.youtube.com/watch?v=_q4aJdbKXKI}
{LaTeX Basics 0½: Emacs - YouTube}

\section{The whole enchilada}
See this document \LaTeX{} source in \href{https://github.com/energos/latex/blob/master/tutorial000.tex}{{\Nerd } GitHub}

% \begin{tcolorbox}
% \begin{verbatim}
% \verbatiminput{this_file_name}
% \end{verbatim}
% \end{tcolorbox}
% \begin{small}
%   \verbatiminput{tutorial000.tex}
% \end{small}

\end{document}
